% =============================================	%
%												%
%		        Dynamic Programming        		%
%												%
% =============================================	%

\chapter{Dynamic Programming}
\chapterinfo{}
	%------------------------------
	\section{Unbounded Knapsack}
	%\sectioninfo{}
	
	Given an infinite amount certain kinds of items, each with a weight and a 
	value, determine the number of each item to include in a collection so that 
	the total weight is less than a given limit and the total value is as large 
	as possible.\\
	\ \\
	{\bf Parameters} : An array of weights, an array of values, and an integer 
	representing the size of the knapsack.\\
	{\bf Returns} : An array of length size+1, the value at index 'i' is the 
	optimal value obtained with weight i. A value of -1 means that weight i is 
	not possible to obtain.\\
	{\bf Complexity} : $O(NT)$ ($N$ = number of items, $T$ = size of knapsack)\\
	
	\lstinputlisting[language=Java,label=samplecode,caption=Unbounded Knapsack (Java)]{Code/unboundedKnapsack.txt}
	
	%------------------------------
	\section{Bounded (0/1) Knapsack}
	%\sectioninfo{}
	
	Given certain kinds of items, each with a weight and a value, determine the 
	largest total value you can take without exceeding your weight limit.\\
    \ \\
    {\bf Note} : You can't take more than one of each item.\\
    {\bf Parameters} : An array in weights, an array of values, and an integer 
    representing the size of the knapsack.\\
    {\bf Returns} : A 2D array of size $size+1 \times knapsack\_size+1$, were the value at index 
    $(i,j)$ is optimal value of a subset of the first $i$ elements that weight a total 
    of $j$. A value of -1 means that it’s impossible to find.\\
    {\bf Complexity} : $O(NM)$ ($N$ = number of items, $T$ = Size of knapsack)\\

    \lstinputlisting[language=Java,label=samplecode,caption=Bounded Knpasack (Java)]{Code/boundedKnapsack.txt}
	
	%------------------------------
	\section{Longest Increasing Subsequence}
	%\sectioninfo{}
	
	{\bf Complexity} : $O(N\log{N})$
    The idea behind this version is:
    Imagine you are playing a game where you have to place a sequence of numbers 
    into stacks. The first number always creates the first stack, then after, 
    you place the rest of the sequence following these rules:
    1.\indent You consider the cards in the same order you received them.
    2.\indent To insert a card in a stack, it must be smaller than all other 
    cards in that stack. Equivalently, it must be smaller than the top of the 
    stack
    3.\indent If there is no stack that can accept a certain card, make a new 
    stack to the right of the last stack.
    4.\indent If there exist multiple stacks which can accept a card, pick the 
    leftmost stack.
    5.\indent Whenever you insert a card into a stack that is not the first, 
    save a pointer from that card to the top of the previous stack.
                                                                                                                      	
    I claim that the number of stacks is the length of the LIS and if you follow 
    the pointers from the very end you can build the LIS.\\
    \ \\                                                                                                           	                                           
    {\bf Note} : It easy to see that the top of all of the stacks will be in increasing 
    order from left to right. (proof via contradiction). Therefore you can find which stack to insert a card 
into in log(k) if you use a binary search, where k is the number of stack.\\

    \lstinputlisting[language=Java,label=samplecode,caption=Longest Increasing Subsequence (Java)]{Code/longestStrictlyIncreasingSubsequence.txt}
	
	%------------------------------
	\section{Longest Common Subsequence}
	%\sectioninfo{}
	
	This algorithm will find the longest common sub-sequence between two 
	strings, $a$ and $b$. $M$ and $N$ are the lengths of $a$ and $b$, 
	respectively. Runs in $O(n^2)$.
	
	\lstinputlisting[language=Java,label=samplecode,caption=Longest Common Subsequence (Java)]{Code/longestCommonSubsequence.txt}
	
	%------------------------------
	\section{Longest Common Substring}
	%\sectioninfo{}
	
	{\bf Complexity} : $O(n^2)$
	
	\lstinputlisting[language=Java,label=samplecode,caption=Longest Common Substring (Java)]{Code/longestCommonSubstring.txt}

	%------------------------------
	\section{Maximum Contiguous Sum}
	%\sectioninfo{}
	
	{\bf Problem Statement} : Find the maximum sum of contiguous elements in an array\\
    {\bf Parameters} : An array of integers\\
    {\bf Returns} : Max sum found\\
    {\bf Complexity} : $O(N)$ where $N$ is the length of the array\\
    
    \lstinputlisting[language=Java,label=samplecode,caption=Maximum Continuous Sum (Java)]{Code/maximumcontiguoussum.txt}

	
	%------------------------------
	\section{Maximum Rectangular Sum}
	%\sectioninfo{}
	
	{\bf Problem Statement} : Finds the maximum sum of all subrectangles in an NxN matrix.
 
    Use a partial sums array on the columns in order to be able to compute the 
    sum of the elements from $g[a][j]$ through $g[b][j]$, for any $a$,$b$ $(a<b)$ 
    in $O(1)$
 
    Then we traverse through all possible horizontal boundaries $y=a$ and $y=b$ and 
    build a 1D array where $array[i]$ = $\sum_{k=a}^{b}g[k][i]$. Then run the 
    maximum contiguous 1D sum algorithm on that array.
	
	%------------------------------
	\section{Levenshtein Distance (Edit Distance)}
	%\sectioninfo{}
	
	The Levenshtein (Edit) Distance is the minimum number of changes in spelling 
	required to change one word into another.
	
	\lstinputlisting[language=Java,label=samplecode,caption=Edit Distance (Java)]{Code/LevenshteinDistance.txt}