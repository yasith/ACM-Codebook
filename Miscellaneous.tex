% =============================================	%
%												%
%		            Miscellaneous               %
%												%
% =============================================	%

\chapter{Miscellaneous}
\chapterinfo{}
	%------------------------------
	\section{Kirchoff's Matrix}
	%\sectioninfo{}
	Kirchoff's matrix allows one to calculate the number of spanning trees in a 
	graph as follows:
	
	1. Compute the Laplacian Matrix as follows\\
	\indent\indent $A[i][i]$ = degree of vertex $i$\\
	\indent\indent $A[i][j]$ = -1 if vertex $i$ and vertex $j$ are connected, 0 otherwise
	
	2. The number of spanning trees is equal to the determinant of any cofactor 
	matrix. (The original matrix with the first row and column removed is a 
	valid cofactor)
	
	%------------------------------
	\section{Josephus Problem}
	%\sectioninfo{}
	
	Of the first $n$ numbers, if you pick and remove the $kth$ number, determine the last 
	one standing.
	
	\lstinputlisting[language=Java,label=samplecode,caption=Josephus Problem (Java)]{Code/josephus.txt}
	
	%------------------------------
	\section{Poker Class}
	%\sectioninfo{}
	
	\lstinputlisting[language=Java,label=samplecode,caption=Poker Class (Java)]{Code/PokerClass.txt}
    
    %------------------------------
	\section{Decimal to Roman Numeral Converter}
	%\sectioninfo{}
	
	\lstinputlisting[language=C++,label=samplecode,caption=Decimal to Roman Numeral Converter (C++)]{Code/Dec2RomanNumeral.txt}

	%------------------------------
	\section{Expression Parsing}
	%\sectioninfo{}
	
	\lstinputlisting[language=C++,label=samplecode,caption=Expression Parsing (C++)]{Code/ExpressionParsing.txt}
